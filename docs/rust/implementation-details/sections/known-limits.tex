\clearpage

\section{已知局限}

\subsection{SwornDisk Linux Rust 实现的局限}

SwornDisk Rust 目前实现的功能仍然是不完整的,仍未完成的功能有:

\begin{itemize}[itemsep=2pt,topsep=0pt,parsep=0pt]
  \item 垃圾回收 (segment cleaning)
  \item 日志
  \item Checkpoint 数据加密
  \item multi logging head
  \item thread logging
\end{itemize}

同时,SwornDisk 仍未经过性能调优,目前比较突出的问题和可能的原因是:

\begin{itemize}[itemsep=2pt,topsep=0pt,parsep=0pt]
  \item 顺序读的性能不符合预期(较低),分析问题出现在从磁盘中读取块花费的时间比较长
\end{itemize}

此外可能还存在若干未发现的缺陷。

\subsection{工程方案上的局限}

尽管目前初步验证了使用 Rust 实现 Linux 内核模块的可行性,但仍然具有很多局限性。例如:

\begin{itemize}
  \item 尽管 rust-for-linux 提供了使用 Rust 编写 Linux 内核模块的模式,但由于现阶段只提供了 “使用 rustc 编译 .rs 文件” 这种程度的支持,不支持使用 cargo 组织工程,对编写 SwornDisk 这样较复杂的内核模块(尤其是在我们还需要向 rust-for-linux 补充能力的情况下)很不友好。
  \item 我们在实现 SwornDisk Linux Rust 的时候尝试引入了 cargo 来组织工程,使用 cargo 的 workspace 来管理多个 crate,实现了分别编译多个依赖 crate 并生成一个目标文件的功能。但这并不代表着可以在其中随意引入第三方 crate,主要原因如下:
  \begin{itemize}
    \item 内核模块不能依赖标准库 (std)
    \item rust-for-linux 对 Rust 语言的核心能力支持不完整,如其只提供 alloc, core 和 kernel 三个 crate,同时 alloc crate 提供的数据结构也是不完整的(例如没有 LinkedList)
    \item 在 rust-for-linux 中,如果尝试在堆上分配数据(如创建 Vec, Box),都需要使用类似 \mintinline{rust}{try_new()}, \mintinline{rust}{try_push()} 这样返回 \mintinline{text}{Result<T>} 的 API 来创建、分配空间,而不能直接使用 \mintinline{text}{new()} 等创建。这些限制会阻碍我们直接使用开源的 crate.
  \end{itemize}
  \item 使用 Rust 编写的 Linux 内核模块,目前只能在同样具有 Rust 支持的 Linux 内核上编译、运行
  \begin{itemize}
    \item 我们尝试过在具有 Rust 支持的内核上编译产生 .ko,复制到没有 Rust 支持的内核加载,会由于内核的 version magic 不同,导致无法加载。
    \item 如果我们跳过对内核模块的 version magic 检查,直接加载 .ko,会由于找不到 Rust 的 alloc, core 和 rust-for-linux 提供的 kernel crate 中方法的符号,无法加载(alloc, core, kernel 是被链接到具有 Rust 支持的内核中的)。
    \item 尽管我们手动将 alloc, core, kernel 这些 crate 编译出的 .o 文件合并到内核模块的 .o 中,参与最终内核模块的生成;但仍然在其它机器上会找不到部分 Linux 内核中的函数的符号(系 kernel crate 使用的 binding)。
    \item 综上,我们目前可以认为,在没有 Rust 支持的内核上加载 Rust 编写的内核模块较难以实现,而为了加载模块必须重新编译具有 Rust 支持的内核的成本也很高。
  \end{itemize}
\end{itemize}