\clearpage
\section{经验体会}

\subsection{如何往 rust-for-linux 里面加东西}

主要包含以下几步:

\begin{itemize}[itemsep=2pt,topsep=0pt,parsep=0pt]
  \item 在 \mintinline{text}{rust/kernel/bindings_helper.h} 引入需要的头文件
  \item make 内核重新生成一下 bindings 文件
  \item 在 \mintinline{text}{rust/kernel} crate 中使用 \mintinline{rust}{bindings::xxx}
  \item 或创建新的 crate ,其中声明 \mintinline{rust}{extern crate kernel} 并使用 \mintinline{rust}{kernel::bindings::xxx}
\end{itemize}

\subsection{使用 Cargo 组织工程}

在 rust-for-linux 中使用 cargo 组织工程,主要需要解决两个问题:

\begin{itemize}[itemsep=2pt,topsep=0pt,parsep=0pt]
  \item 在 cargo 调用 rustc 编译的时候,需要添加一系列参数。这部分参数通过 \mintinline{text}{.cargo/config.toml} 声明;由于需要引用内核代码树中的文件,需要注意路径。
  \item cargo 工程类型需要是 rlib, 同时生成的文件需要是 .o 格式的目标文件 (emit=objs);多个 crate 生成的 .o 文件需要使用 ld.lld 合并成一个 .o 文件。
  \item 需要写一个 shell 脚本,给生成的 .o 文件创建一个 .cmd 格式的声明文件,声明该模块的依赖路径和源码路径等(参考 \mintinline{text}{dm-sworndisk/scripts/generate_cmd.sh})。
  \item 接下来的步骤 (modpost, lto, ...) 交给 Linux Kbuild 完成即可。
\end{itemize}

\subsection{如何在 Rust 的 SwornDisk 中管理全局变量}

在实现 SwornDisk 的时候免不了需要用到一些全局共享的东西,由于 rust-for-linux 的诸多限制,所以我们不能用 \mintinline{text}{lazy_static} 来声明全局变量(主要原因是它依赖了 spin 这个 crate 模拟并发原语)。要在全局范围内访问一些内容,主要有两种解决方案:

\begin{itemize}[itemsep=2pt,topsep=0pt,parsep=0pt]
  \item 将值分配在堆上,指针交给某些具有 private 成员的结构体保存。
  \item 使用 \mintinline{rust}{Box<Option<T>>} 结构 (unsafe).
\end{itemize}

\clearpage
\subsection{一些常见的场景的实现}

\subsubsection{如何加解密一个块}

\begin{minted}{rust}
use crypto::{Aead, get_random_bytes};

let mut aead = Aead::new(c_str!("gcm(aes)"), 0, 0).unwrap();

let key = get_random_bytes(16);         // Vec<u8>
let mut nonce = get_random_bytes(12);   // Vec<u8>
let mut plain = get_random_bytes(4096); // Vec<u8>

// 加密
let (mut cipher, mut mac) = aead.as_ref()
  .encrypt(&key, &mut nonce, &mut plain,)
  .unwrap();

// 解密
let plain = aead.as_ref()
  .decrypt(&key, &mut mac, &mut nonce, &mut cipher)
  .unwrap();
\end{minted}

\subsubsection{如何从硬盘读写一个块}

\begin{minted}{rust}
// 分配一个块
let mut block = Vec::new();
block.try_resize(BLOCK_SIZE as usize, 0u8)?;

// 创建一个 DmIoRegion, 注意大小以扇区为单位
// DmIoRegion::new(block_device, sector, nr_sectors) -> Result<DmIoRegion>
let mut region = DmIoRegion::new(&bdev, record.hba, BLOCK_SECTORS)?;

// 创建 Device Mapper I/O 请求
let mut io_req = DmIoRequest::with_kernel_memory(
    READ as i32,                   // 写入请求时此处为 WRITE
    READ as i32,
    block.as_mut_ptr() as *mut c_void,
    0,
    client,                        // DmIoClient
);

// 提交 IO 请求
io_req.submit(&mut region);
\end{minted}

\subsubsection{如何将一个 struct 储存到硬盘上}

由于 rust-for-linux 中不能用标准库,因此也不能用 \mintinline{rust}{io::Read} 和 \mintinline{rust}{io::Write} 这种 trait,此时如何把一个 Rust struct 持久化地保存到硬盘上就成为一个问题。

我们采用的方案是将块二进制序列化,通过实现 \mintinline{rust}{Serialize} 和 \mintinline{rust}{Deserialze} trait 可以将一个 struct 序列化为二进制串(或从二进制串解析出 struct 本身):

\begin{minted}{rust}
/// Serailize trait: convert a struct into binary bufferr (Vec<u8>)
pub trait Serialize {
    fn serialize(&self) -> Result<Vec<u8>>;
}

/// Deserialize trait: convert a binary buffer (&Vec<u8>) into a struct
pub trait Deserialize {
    fn deserialize(buffer: &[u8]) -> Result<Self>
    where
        Self: Sized;
}
\end{minted}

PS: 其实我曾经想尝试一下引入 serde, 不过看起来很麻烦,当时安排比较紧没有那么多时间可以尝试。

\subsubsection{如何创建一个异步任务 (worker) 并加入队列中}

\begin{minted}{rust}
use cmwq::{WorkFuncTrait, WorkQueue, WorkStruct};

struct Worker;

impl WorkFuncTrait for Worker {
  fn work(_work_struct: *mut bindings::work_struct) -> Result {
    // do something...
    // *mut bindings::work_struct can be used for calling `container_of!()`
  }
}

let mut work_queue = WorkQueue::new(c_str!("queue"), bindings::WQ_UNBOUND | bindings::WQ_MEM_RECLAIM, 0)?;

let mut worker = WorkStruct::new();
worker.init::<Worker>();

work_queue.queue_woork(&mut worker);

\end{minted}