\clearpage
\section{编译、运行、测试}

\subsection{编译 rust-for-linux}

参考 \href{https://github.com/Rust-for-Linux/linux/blob/rust/Documentation/rust/quick-start.rst}{rust-for-linux Quick Start 文档},编译具有 Rust 支持的内核。

根据使用的发行版不同,编译的方法也可能不同。以 Arch Linux 为例,首先需要安装依赖:

\begin{minted}{bash}
$ sudo pacman -S base-devel clang lld python3 llvm bc cpio
\end{minted}

安装 rustup 工具链:

\begin{minted}{bash}
$ curl --proto '=https' --tlsv1.2 -sSf https://sh.rustup.rs | sh
\end{minted}

克隆 rust-for-linux:

\begin{minted}{bash}
$ git clone git@github.com:occlum/sworndisk-linux-rs.git linux
$ cd linux
\end{minted}

根据 rust-for-linux 的要求设置 Rust 环境,安装必要的组件:

\begin{minted}{bash}
$ rustup override set $(scripts/min-tool-version.sh rustc)
$ rustup component add rust-src
$ cargo install --locked --version $(scripts/min-tool-version.sh bindgen) bindgen

$ rustup component add rustfmt
$ rustup component add clippy
\end{minted}

\textbf{PS: 由于 rust-for-linux 项目强制使用 nightly 版本的工具链,SwornDisk 使用的 Rust 版本为 rustc 1.60.0-nightly (9ad5d82f8 2022-01-18),在安装 Rust 工具链时,请务必手动切换到此版本:}

\begin{minted}{bash}
$ rustup toolchain install nightly-2022-01-18
\end{minted}

导出当前系统的内核配置:

\begin{minted}{bash}
$ zcat /proc/config.gz > .config
\end{minted}

编辑 \mintinline{text}{.config} 启用以下选项:

\begin{minted}{text}
CONFIG_RUST_IS_AVAILABLE=y 
CONFIG_RUST=y 
CONFIG_DM_PERSISTENT_DATA=y 
CONFIG_DM_BUFIO=y 
CONFIG_LIBCRC32C=y 
CONFIG_BLK_DEV_LOOP=y 
CONFIG_BLK_DEV_DM=y 
CONFIG_BLK_DEV_LOOP_MIN_COUNT=8
\end{minted}

\textbf{PS: 部分选项可能由于依赖或其它各种原因,在 \mintinline{text}{.config} 中无法配置,此时可以修改 \mintinline{text}{include/config/auto.conf} 文件。}

编译内核:

\begin{minted}{bash}
$ make LLVM=1 -j8
$ sudo make modules_install
\end{minted}

参考 \href{https://wiki.archlinux.org/title/Kernel_(%E7%AE%80%E4%BD%93%E4%B8%AD%E6%96%87)/Traditional_compilation_(%E7%AE%80%E4%BD%93%E4%B8%AD%E6%96%87)}{Kernel - ArchWiki} 更新 initcpio 和 grub,重启即可。

附:\href{https://kirainmoe.feishu.cn/wiki/wikcnyStut2uUsg0RBWGwtOoaPc}{Ubuntu 下编译 rust-for-linux 的过程记录}供参考。

\subsection{编译 SwornDisk}

SwornDisk Rust 源码位于 \mintinline{text}{modules/sworndisk} 中。

\begin{minted}{bash}
$ cd modules/sworndisk
$ make clean
$ make
\end{minted}

若一切正常,应当得到 SwornDisk Linux 内核模块 \mintinline{text}{dm-sworndisk.ko}:

\begin{minted}{bash}
$ modinfo dm-sworndisk.ko

filename:       /home/bellaris/Workspace/linux/modules/sworndisk/dm-sworndisk.ko
author:         Occlum Team
description:    Rust implementation of SwornDisk based on Linux device mapper.
license:        GPL v2
vermagic:       5.17.0-rc8-126275-g6b600e79f6e6-dirty SMP preempt mod_unload 
name:           dm_sworndisk
retpoline:      Y
depends:        
srcversion:     4BAC5027D1352F0DF2B3A7E
parm:           run_unittest:Run dm-sworndisk kernel module unit test (bool)
\end{minted}

\subsection{加载并创建 SwornDisk}

首先加载 SwornDisk 内核模块:

\begin{minted}{bash}
$ sudo insmod dm-sworndisk.ko
\end{minted}

SwornDisk Linux Rust 需要挂载两个物理设备分区(数据设备、元信息设备),我们用 \mintinline{text}{dd} 创建空磁盘文件,使用 \mintinline{text}{losetup} 挂载为回环设备:

\begin{minted}{bash}
$ dd if=/dev/null of=~/tmp/disk.img seek=58593750   # 30GB Data
$ dd if=/dev/null of=~/tmp/meta.img seek=8388608    # 4GB Meta
$ sudo losetup /dev/loop0 ~/tmp/disk.img
$ sudo losetup /dev/loop1 ~/tmp/meta.img
\end{minted}

使用 \mintinline{text}{dmsetup} 创建 SwornDisk Device Mapper 目标设备:

\begin{minted}{bash}
$ echo -e '0 58593750 sworndisk /dev/loop0 /dev/loop1 0 force' | sudo dmsetup create test-sworndisk
\end{minted}

命令用法:

\begin{minted}{bash}
echo -e '0 <size> sworndisk <data_dev> <meta_dev> 0 force' | sudo dmsetup create <name>
\end{minted}

\begin{itemize}[itemsep=2pt,topsep=0pt,parsep=0pt]
  \item \mintinline{text}{<size>}: 磁盘扇区数量,扇区大小为 512B
  \item \mintinline{text}{<data_dev>}: 数据磁盘对应设备文件
  \item \mintinline{text}{<meta_dev>}: 元数据磁盘对应设备文件
  \item \mintinline{text}{<format>}: 是否格式化创建磁盘:(force: 强制格式化创建新磁盘, true: 损坏时格式化, false: 不格式化)
  \item \mintinline{text}{<name>}: 磁盘名称
\end{itemize}

此时成功创建了一个名为 \mintinline{text}{test-sworndisk} 的虚拟块设备,位于 \mintinline{text}{/dev/mapper/test-sworndisk}.

\subsection{测试}

\subsubsection{fio 性能测试}

\mintinline{text}{scripts/fio.conf} 中定义了 fio 测试的配置。

\begin{minted}{bash}
$ sudo fio scripts/fio.conf
\end{minted}

\subsubsection{单元测试}

由于使用 Rust 编写的 Linux 内核模块无法直接使用 \mintinline{text}{cargo test} 进行单元测试,因此单独写了一个模块 \mintinline{text}{unitest} 实现单元测试。在加载内核模块时带参数 \mintinline{text}{run_unittest=true} 即可。

\begin{minted}{bash}
$ sudo insmod dm-sworndisk.ko run_unittest=true
\end{minted}